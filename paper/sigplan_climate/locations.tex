%% \section{Data Analysis: Locations}

%% \subsection{How Carbon Efficient Were These Conferences?}

%% Present the results using a couple of metrics

%% \subsection{How Did the Conferences Do in Attracting Local Participants?}

%% Show 1st quartile and median distances

%% \subsection{Were There Geographic Differences Between the Four Conferences?}

%% POPL/ICFP (Europe-centered) vs PLDI/SPLASH (NA-centered)

%% \subsection{What Would Have Been the Ideal Locations for Each Conference?}

\section{A Retrospective Speculation: Picking the Optimal Destination for Past Conferences}
\label{sec:speculate}
%% \paragraph{A naive, retrospective, optimal choice}

\cvl{Try to do this in Google Earth. Caveat: this is a non-linear system,
  because the participants depend partly on the location... so that needs to
  be accounted for.}\bcp{I think this bit with Google Earth doesn't have to
  be done now?} 

We have observed that the location an \event takes place in significantly
impacts the distribution of origin of its participants. However, setting
this factor aside temporarily to consider what could have been the cheapest
location for past conferences, assuming that the change in location would
cause no change in participants, can be an illuminating exercise.

To this end, we chose a fixed number of locations that we believe to be
representative and spread across the relevant parts of the globe: Paris,
Edinburgh, Boston, Los Angeles, Vancouver, Tokyo, Beijing, and Mumbai. We
then reprocessed the dataset to look for the location in this set that would
have led to the lowest carbon footprint for each event, assuming that it
would not have changed the set of participants.

\begin{table}
  \begin{tabular}{|l|l|c|c|c|c|}
    \hline%
    \bfseries Event & \bfseries Orig. Loc. & \bfseries Orig. Cost & \bfseries Best Loc. & \bfseries Best Cost & \bfseries Saved
    \csvreader[head to column names]{../../output/sigplan/optimal_loc.csv}{}%
              {\\\conf\ \year & \csvcoliii & \csvcoliv & \csvcolv & \csvcolvi & \csvcolvii}%
              \\\hline
  \end{tabular}
  \caption{For each \event, depicts the location among Paris, Edinburgh,
    Boston, Philadelphia, Los Angeles, Vancouver, Tokyo, Beijing and Mumbai
    \bcp{and Copenhagen, Gothenburg, Oxford, ...} that would
    have led to the lowest carbon footprint. Starred best locations indicates that they coincide with the original one.
    The final column shows the amount
    of \gazunitbis that it would have saved.\bcp{in what units?} \bcp{Could
      we display 0.0 as blank?}}
  \label{table:optimal}
\end{table}

Figure~\ref{table:optimal} depicts the resulting data: for each \event, the
best location, and the average \gazunitbis{} it would have saved. We observe
that in the majority of the events, the locality effect is strong enough
that the optimal location is on the same continent as the actual
location. However, it is striking to see how often the east coast of the US
turns out to be the cheapest destination. In particular, it appears to be
preferable to the west coast in most cases (in spite of the underlying
locality effect that we are ignoring here\bcp{don't remember what we meant
  by this}).

\begin{obs}
Due to the locality effect, past data can act as a heuristic for a worst
case distribution of attendance with respect to the objective function of
minimizing the carbon footprint. Doing so most notably suggests that the
east coast of the US is generally a lower-carbon location than the west
coast for this group of conferences.
  \label{obs:optim}
\end{obs}
