\section{Data Analysis: Locations}

\subsection{How Carbon Efficient Were These Conferences?}

Present the results using a couple of metrics

\subsection{How Did the Conferences Do in Attracting Local Participants?}

Show 1st quartile and median distances

\subsection{Were There Geographic Differences Between the Four Conferences?}

POPL/ICFP (Europe-centered) vs PLDI/SPLASH (NA-centered)

\subsection{What Would Have Been the Ideal Locations for Each Conference?}

\cvl{Try to do this in Google Earth. Caveat: this is a non-linear system, because the participants depend partly on the location... so that needs to be accounted for.}

\paragraph{A naive, retrospective, optimal choice}

We have observed that the location an \event takes place in significantly
impacts the distribution of origin of its participants. However, dismissing
temporarily this factor to consider what could have been the cheapest location
for past conferences can be an illuminating exercise.

To this end, we chose a fixed number of locations that we believe to be
representative and spread across the relevant parts of the globe for our
concern: Paris, Edinburgh, Boston, Los Angeles, Vancouver, Tokyo, Beijing and
Mumbai. We then reprocessed the dataset to look for the location that would have
led to the lowest carbon footprint, once again assuming that it would not have
changed the set of participants.

\begin{table}
  \begin{tabular}{|l|l|c|c|c|c|}
    \hline%
    \bfseries Event & \bfseries Orig. Loc. & \bfseries Orig. Cost & \bfseries Best Loc. & \bfseries Best Cost & \bfseries Saved
    \csvreader[head to column names]{../../output/sigplan/optimals.csv}{}%
              {\\\conf\ \year & \csvcoliii & \csvcoliv & \csvcolv & \csvcolvi & \csvcolvii}%
              \\\hline
  \end{tabular}
  \caption{For each \event, depicts the location among Paris, Edinburgh,
    Boston, Philadelphia, Los Angeles, Vancouver, Tokyo, Beijing and Mumbai that would
    have led to the lowest carbon footprint. Starred best locations indicates that they coincide with the original one.
    The final column shows the amount
    of \gazunitbis that it would have saved.}
  \label{table:optimal}
\end{table}

Figure~\ref{table:optimal} depicts the resulting data: for each \event, the best location
and the average \gazunitbis it would have saved. We observe that in the majority
of the events, the locality effect is strong enough that the optimal takes place on
the original continent. However it is striking to see how often the East Coast turns
out to be the cheapest destination. In particular it appears to be preferable to
the West Coast in most cases, even despite the underlying locality effect that
is ignored here.

\begin{obs}
  Due to the locality effect, past data can act as a heuristic for a worst case
  distribution of attendance with respect to the objective function of
  minimizing the carbon footprint. Doing so most notably suggests that the East
  Coast should often be preferred to the West Coast.
  \label{obs:optim}
\end{obs}

