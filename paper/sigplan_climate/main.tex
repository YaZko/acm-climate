\documentclass[manuscript, review, screen]{acmart}

\citestyle{acmauthoryear}

\acmVolume{0}
\acmNumber{0}
\acmArticle{0}
\acmYear{2019}
\acmMonth{0}

\setcopyright{acmcopyright}

\acmDOI{0000001.0000001}

\usepackage[english]{babel}
\usepackage[utf8x]{inputenc}

\usepackage{caption}
\usepackage{subcaption}

\usepackage{amsfonts} % mathbb
\usepackage{amsthm}   % environment proof
\usepackage{stmaryrd} % semantical brackets

\usepackage{siunitx} 

\usepackage{xcolor}
%% \usepackage[hidelinks]{hyperref}
\newif\ifcomments\commentstrue   %% include author discussion

\usepackage{glossaries}
\usepackage{csvsimple}

\ifcomments
\newcommand{\yz}[1]{\textcolor{blue}{{[YZ:~#1]}}}
\newcommand{\bcp}[1]{\textcolor{red}{{[BCP:~#1]}}}
\newcommand{\cvl}[1]{\textcolor{green}{{[CVL:~#1]}}}
\else
\newcommand{\yz}[1]{}
\newcommand{\bcp}[1]{}
\newcommand{\cvl}[1]{}
\fi

\newcommand{\python}{\texttt{Python 3}}
\newcommand{\gaz}{\si{\kilogram\of{CO_2e}}}
\newcommand{\gazunit}{\si{\kilogram\of{CO_2e~ per~ passenger}}}
\newcommand{\gazunitbis}{\si{\tonne\of{CO_2e}}}

\newcommand{\event}{event} % What's the right term to designate a specific iteration of a conference?
\newcommand{\conf}{conference} 

\newtheorem{obs}{Observation}
\newtheorem{recommend}{Recommendation}

\title{Engaging with Climate Change: grounding in data possible changes for SIGPLAN}

\begin{document}

\title{Towards Sustainable Conferences: An Analysis of SIGPLAN Conference Data}

\author{Cristina V. Lopes}
%\orcid{1234-5678-9012-3456}
\affiliation{%
  \institution{University of California, Irvine}
  \city{Irvine}
  \state{CA}
  \postcode{92697}
  \country{USA}}
\email{lopes@uci.edu}

\author{Benjamin Pierce}
\affiliation{%
  \institution{University of Pennsylvania}
  \city{Philadelphia}
  \state{PA}
  \country{USA}}
\email{bcpierce@cis.upenn.edu}

\author{Yannick Zakowski}
\affiliation{%
  \institution{University of Pennsylvania}
  \city{Philadelphia}
  \state{PA}
  \country{USA}}
\email{zakowski@seas.upenn.edu}

\begin{abstract}

Starting in 2017, SIGPLAN formed an ad-hoc committee to study issues related to
climate change, in particular how SIGPLAN can improve its operations in order to
contribute to 40\% reduction in carbon emissions by 2030. One important part of
that effort was to gather data pertaining to SIGPLAN conferences so to gain a
better understanding of how we have been operating so far. This paper explain
the data we gathered, and presents an analysis of this data. The main conclusion
is that there is an inherent conflict between SIGPLAN's mission of geographic
inclusion and carbon emissions, and that, going forward, innovative approaches
for how to organize conferences that are both inclusive and carbon efficient
will be needed.

\end{abstract}

\keywords{Climate change, conferences, carbon footprint}

\maketitle

\section{Introduction}

Given the existential threat of global warming, it is incumbent on
individuals and organizations to evaluate the carbon emissions
associated with their activities and find ways to reduce them.  For many
academic researchers, these emissions will overwhelmingly come from air
travel, especially to international conferences.

This observation raises a number of questions about how to organize our
professional activities so as to maximize progress while minimizing
emissions.  Should SIGPLAN conference locations be chosen to minimize their
carbon impact? If
so, how? Should we move toward co-locating conferences? Or, on the contrary,
should some conferences be split into regional meetings or held
simultaneously at two sites on different continents?
Should we continue holding some conferences entirely virtually,
post Covid?
% \bcp{This is a
%   good question, of course, but I'm not sure it is addressed at all by the
%   data we have.}
% \yz{My rationale was that all these questions require hard numbers to ponder
%   correctly, which is the object of this paper. None of these questions should
%   really find an answer here, should it?}

To ground discussions about the decisions and compromises that the
scientific community may collectively wish to undertake, at least three
main sorts of data seem useful.
\begin{itemize}
\item The estimated emissions of past conferences.
\item The geographical distribution of participants to conferences.
\item The overlap in participation between various conferences.
\end{itemize}

We outline the results of a preliminary analysis of the past several
years of registration data for four of the main SIGPLAN conferences.  We
hope this effort can serve as a basis both for debates about concrete
measures and for larger and more comprehensive studies.

After briefly describing our dataset in Section~\ref{sec:dataset}, we
present estimates of the individual footprints of each conference in
Section~\ref{sec:footprint}.  In Section~\ref{sec:community}, the core of
our analysis, we derive several statistics about the geographical
distribution of participants and their habits of
cross-participation---across years and across conferences---arguing that
these data are correlated to the footprint.  We then present in
Section~\ref{sec:speculate} a speculative experiment aiming to estimate
``ideal locations'' for past conferences in order to minimize their
footprints.  \ifopinions In Section~\ref{sec:opinions} we draw some concrete
recommendations for future conference organizers based on these analyses.
\fi Finally, in Section~\ref{sec:software}, we outline the open-source tool
we developed to conduct our analyses, in hopes that other communities might
piggyback on our efforts to conduct their own similar studies.

\section{Dataset}
\label{sec:dataset}
 
Our dataset consists of 10 years worth of attendance to the four major
SIGPLAN conference series---POPL, PLDI, ICFP, and SPLASH---from the
beginning of 2009 until the end of 2018. Data for some of the conferences in the
earlier years are missing. In total, we have data for 33 conferences,
corresponding to 8,758 unique participants and 16,374 trips. For each
participant, we know all the conferences (s)he attended, and from which city
(s)he departed to attend the conferences.
%
The names of participants are replaced in the dataset by unique hashes,
obscuring each individual's identity while allowing them to be identified
across years and across the conferences they attended.



\section{The footprint of conferences: an estimation}
\label{sec:footprint}

Carbon footprint is the essential metric of interest that we seek to reduce.
As such, it is also the starting point of our analysis.
We introduce in this section the methodology we used and tool we built to
conduct all of our analyses, and describe the first results ran on our dataset.

\subsection{Evaluation of the carbon cost: methodology}
\label{sec:methodo}

We conduct all our analyses through a \python{} script, publicly available at
\url{https://github.com/YaZko/acm-climate}. We describe its behavior and give
a brief overview of its use in Section~\ref{sec:software}.
\bcp{Maybe this repo needs a more
informative name?}

For the carbon footprint estimate we present in this paper,
it is relevant to keep in mind the following assumptions that have
been made:
\begin{itemize}
\item we assume that \emph{all} participant travelled by plane;
\item we assume that the airports used are close enough to the end points for their locations to be assimilated;
\item we assume that all flights are direct flights;
\item we assume that the geodesic distance is the one taken by planes.
\end{itemize}

Estimating the errors introduced by these hypotheses and refining them would be
a valuable work. However, for this first proposal that mainly aims at a relative
evaluation of different structural changes in our activities, we believe those
to be workable hypotheses.

The distance traveled by the participants is converted to an emission expressed
in \gaz. To do so, we use a standard model introduced as part of the
\texttt{DEFRA 16} report on Greenhouse gas
\footnote{\url{https://www.gov.uk/government/publications/greenhouse-gas-reporting-conversion-factors-2016}}
\footnote{\url{https://co2calculator.acm.org/methodology.pdf}}
conducted by the British Government.

The model distinguishes three classes of flight, depending on their length:
short, medium or long hauls. Each category is associated with a linear
coefficient relating directly the distance of travel to the amount of \gaz
emitted. We make the assumptions, anecdotally observed, that researchers all
fly in economy class.

A second linear coefficient, identical for all flights and so-called
\emph{radiative forcing index}, is added to account for the difference in
radiative forcing between the same emissions at ground level compared to high in
the atmosphere. This coefficient has been taken for the results presented in this
paper to be $1.891$ as suggested by R. Sausen et al.~\cite{Sausen05}

We therefore obtain the following piece-wise linear model of emissions for a flight covering $d$ kms:
\begin{itemize}
\item $1.891 * 0.14735 * d$ \gazunit if $d < 785$
\item $1.891 * 0.08728 * d$ \gazunit if $785 \leq d < 3700$
\item $1.891 * 0.077610 * d$ \gazunit if $3700\leq d  $
\end{itemize}

%% It should be noted that experiments with other models show significant variance
%% in absolute value, but resilience in relative values.\bcp{Maybe worth
%%   showing some numbers justifying these statements?}\yz{I agree, will
%%   do}\bcp{Assuming that we can get our numbers to agree with CoolEffect's,
%%   we could also mention this!} Once again, refining the
%% model would hence be a valuable work, but using this simple standard and
%% well-established one appears appropriate to draw conclusion in terms of
%% \emph{relative} impact of different measures.

%% This first pass of the script therefore give us an estimation of the footprint
%% of our conferences. We have implemented on top of it several analyses aiming to
%% estimate the correlation some concrete factors upon which conference organizers
%% can act may have with this footprint.
%% The description of these analyses will cover Section~\ref{sec:community} to \ref{sec:speculate}.

\subsection{Footprint of the studied conferences}

\begin{table}
\begin{tabular}{|l|l|c|c|c|}
  \hline%
  \bfseries Event & \bfseries Location & \bfseries \# Participants & \bfseries Total cost & \bfseries Average cost 
\csvreader[head to column names]{../../output/sigplan/footprint_confs.csv}{}%
{\\\conf\ \year & \location & \csvcoliv & \csvcolv & \csvcolvi}%
\\\hline
\end{tabular}
\caption{For each \event: location, number of participants and carbon cost, total and average per participant, in \gazunitbis,}
\label{table:footprint}
\end{table}

We now turn to the estimation of the footprint of our dataset.
Table~\ref{table:footprint} depicts the total and average carbon cost per participant of
all conferences analyzed. This cost is estimated in terms of \gazunitbis of emissions.
The main data of interest is arguably the last column depicting the average cost per participant.

This high level data already suggests an interesting observation:
\begin{obs}
If the carbon footprint of conferences due to air travel is indeed significant,
its specific average value per participant varies from an \event to another by up
to a factor of 2.
\label{obs:footprint}
\end{obs}

Indeed, the lowest average carbon cost of our dataset is obtained by PLDI'18 at 0.9\gazunitbis,
while the highest one is reached by ICFP'16 at 1.93\gazunitbis.
Understanding the underlying rationals for such variations may be a promising
angle to reduce our emissions without restructuring fundamentally our activity.

\input{}
\input{}
\input{}

\end{document}
