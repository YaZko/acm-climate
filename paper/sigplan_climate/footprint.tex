\section{The footprint of conferences: an estimation}

Carbon footprint is the essential metric of interest that we seek to reduce.
As such, it is also the starting point of our analysis.
We introduce in this section the methodology we used and tool we built to
conduct all of our analyses, and describe the first results ran on our dataset.

\subsection{Evaluation of the carbon cost: methodology}
\label{sec:methodo}

\yz{This should probably be cut from the paper. Entirely, or abstracted?}

We conduct all our analyses through a \python script, publicly available at
\url{https://github.com/YaZko/acm-climate}.

The script requires as an input a dataset similar to ours, described by two
\texttt{csv} files. The first one describes the list of conferences: each line
describes a specific event and the location it took place in, i.e. has the
fields \texttt{Name, Year, City, State} and \texttt{Country}. The second one
contains the list of participants of these events: each line describes a unique
participation at an event with the location of origin of the participant, i.e.
has the fields \texttt{Identifier, City, State, Country, Conference} and
\texttt{Year}.

The first pass of the analysis computes the needed raw data. 
Informal named locations manually provided by participant are mapped to their
ISO designation using the \texttt{pycountry} library.
Once this is done, these named locations are converted to GPS
locations using the \texttt{geopy} library, that provides a straightforward api
to this end.

%% An open source data-base of airports and their locations is then used to
%% manually find the closest one to each city of concern, among those classified as
%% \emph{medium} or \emph{large}.

Distances in kilometers between airports are then computed between GPS locations
once again using the \texttt{geopy} library. They use the geodesic distance
(shortest distance for an ellipsoidal model of the Earth) with a model providing
precisions that are several orders more precise than we need.

At this point, we therefore know for each participant to a conference the
distance they traveled.
%% the two airports between which he is most likely to have commuted.
It is relevant to keep in mind the following assumptions that have
been made to reach this stage:
\begin{itemize}
\item we assume that \emph{all} participant travelled by plane;
\item we assume that the airports used are close enough to the end points for their locations to be assimilated;
\item we assume that all flights are direct flights;
\item we assume that the geodesic distance is the one taken by planes.
\end{itemize}

Estimating the errors introduced by these hypotheses and refining them would be
a valuable work. However, for this first proposal that mainly aims at a relative
evaluation of different structural changes in our activities, we believe those
to be workable hypotheses.

Remains now to convert this distance to an emission in \gaz. We use to this end
a standard model introduced as part of the \texttt{DEFRA 16} report on
Greenhouse gas
\footnote{\url{https://www.gov.uk/government/publications/greenhouse-gas-reporting-conversion-factors-2016}}
\footnote{\url{https://co2calculator.acm.org/methodology.pdf}}
conducted by the British Government.

The model distinguishes three classes of flight, depending on their length:
short, medium or long hauls. Each category is associated with a linear
coefficient relating directly the distance of travel to the amount of \gaz
emitted. We make the assumptions, anecdotally observed, that researchers all
fly in economy class.

A second linear coefficient, identical for all flights and so-called
\emph{radiative forcing index}, is added to account for the difference in
radiative forcing between the same emissions at ground level compared to high in
the atmosphere. This coefficient has been taken for the results presented in this
paper to be $1.891$ as suggested by R. Sausen et al.\cite{Sausen05}.

We therefore obtain the following piece-wise linear model of emissions for a flight covering $d$ kms:
\begin{itemize}
\item $1.891 * 0.14735 * d$ \gazunit if $d < 785$
\item $1.891 * 0.08728 * d$ \gazunit if $785 \leq d < 3700$
\item $1.891 * 0.077610 * d$ \gazunit if $3700\leq d  $
\end{itemize}

It should be noted that experiments with other models show significant variance
in absolute value, but resilience in relative values.\bcp{Maybe worth
  showing some numbers justifying these statements?}\yz{I agree, will
  do}\bcp{Assuming that we can get our numbers to agree with CoolEffect's,
  we could also mention this!} Once again, refining the
model would hence be a valuable work, but using this simple standard and
well-established one appears appropriate to draw conclusion in terms of
\emph{relative} impact of different measures.

This first pass of the script therefore give us an estimation of the footprint
of our conferences. We have implemented on top of it several analyses aiming to
estimate the correlation some concrete factors upon which conference organizers
can act may have with this footprint.
The description of these analyses will cover Section~\ref{sec:community} to \ref{sec:speculate}.

\subsection{Footprint of the studied conferences}

\begin{table}
\begin{tabular}{|l|l|c|c|c|}
  \hline%
  \bfseries Event & \bfseries Location & \bfseries \# Participants & \bfseries Total cost & \bfseries Average cost 
\csvreader[head to column names]{../../output/sigplan/footprint_confs.csv}{}%
{\\\conf\ \year & \location & \csvcoliv & \csvcolv & \csvcolvi}%
\\\hline
\end{tabular}
\caption{For each \event: location, number of participants and carbon cost, total and average per participant, in \gazunitbis,}
\label{table:footprint}
\end{table}

We now turn to the estimation of the footprint of our dataset.
Figure~\ref{table:footprint} depicts the total and average carbon cost per participant of
all conferences analyzed. This cost is estimated in terms of \gazunitbis of emissions.
The main data of interest is arguably the last column depicting the average cost per participant.

This high level data already suggests an interesting observation:
\begin{obs}
If the carbon footprint of conferences due to air travel is indeed significant,
its specific average value per participant varies from an \event to another by up
to a factor of 2.
\label{obs:footprint}
\end{obs}

Indeed, the lowest average carbon cost of our dataset is obtained by PLDI'18 at 0.9\gazunitbis,
while the highest one is reached by ICFP'16 at 1.93\gazunitbis.
Understanding the underlying rationals for such variations may be a promising
angle to reduce our emissions without restructuring fundamentally our activity.
