\section{Introduction}

In face of the global warming threat, we are responsible to evaluate the carbon
emissions entailed by our activity, and seek for ways to reduce this footprint.
It is well-known that the overwhelming factor of emission in research is air travel,
notably due to international conferences.

With this perspective in mind, we may be attracted to ponder several questions
about our activity. Should SIGPLAN conference locations be chosen to minimize
their carbon impact? Should we move toward co-locating more conferences? Or on
the contrary, should some conferences be split into two remote sites?
More drastically, should some conferences be held entirely virtually?
% \bcp{This is a
%   good question, of course, but I'm not sure it is addressed at all by the
%   data we have.}
% \yz{My rationale was that all these questions require hard numbers to ponder
%   correctly, which is the object of this paper. None of these questions should
%   really find an answer here, should it?}

Such decisions have various impacts on our activity, and may have drawbacks in
terms of the social and professional benefits of conferences. 
In order to ground discussions about the compromises we wish to collectively undertake, we consider
three main classes of data whose analysis may guide the decisions of the community.
\begin{itemize}
\item What has been the emissions of past conferences?
\item What is the geographical distribution of participants to conferences?
\item What is the overlapping in participation that exists between various conferences?
\end{itemize}

We propose in this paper one such analysis that we hope can be seen as a basis both for
debates about concrete measures, as well as the basis for more involved data analyses.

\bcp{This paragraph needs an update:}Section~\ref{sec:dataset} and
\ref{sec:methodo} describe respectively the analyzed data set and the
methodology followed. The aggregated data we derived is then exposed through
Section~\ref{sec:data}: we point a few observations of interests, but aim to
remain neutral through this exposition.  Finally, we offer the conclusions
we derive from the data and the measures we correspondingly advice the ACM
to take through Section~\ref{sec:opinions}.

