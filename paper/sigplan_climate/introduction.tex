\bcp{We should remove the notices below that say the manuscript has been
  submitted to ACM, and the copyright and permissions stuff.}

\section{Introduction}

Given the existential threat of global warming, it is incumbent on every
individual and organization to evaluate the carbon emissions footprint
generated by their activities and consider ways to reduce them.  For many
academic researchers, this footprint will overwhelmingly come from air
travel, especially to international conferences.

This observation raises a number of questions about how to organize our
activities so as to maximize progress while minimizing emissions.  Should
SIGPLAN conference locations be chosen to minimize their carbon impact? If
so, how? Should we move toward co-locating conferences? Or, on the contrary,
should some conferences be split into regional meetings or held
simultaneously at two sites on different continents?  More drastically,
do we need to hold some conferences entirely virtually?
% \bcp{This is a
%   good question, of course, but I'm not sure it is addressed at all by the
%   data we have.}
% \yz{My rationale was that all these questions require hard numbers to ponder
%   correctly, which is the object of this paper. None of these questions should
%   really find an answer here, should it?}

To ground discussions about the decisions and compromises that the
scientific community may collectively wish to undertake, we consider three
main classes of data whose analysis may guide our decisions.
\begin{itemize}
\item The estimated emissions of past conferences.
\item The geographical distribution of participants to conferences.
\item The overlap in participation between various conferences.
\end{itemize}

We outline the results of a preliminary analysis effort on the past several
years of registration data for four of the main SIGPLAN conferences.  We
hope this effort can server as a basis both for debates about concrete
measures as well as for larger scale and farther reaching studies.  

After briefly describing our dataset in Section~\ref{sec:dataset}, we
present an estimate of the SIGPLAN conferences' footprint in
Section~\ref{sec:footprint}.  Section~\ref{sec:community} is the core of our
analysis: we derive several statistics about the geographical distribution
of the participants and their habits of cross participation, across time and
across conferences, arguing that these data are correlated to the footprint.
We then present in Section~\ref{sec:speculate} a speculative experiment
aiming to estimate the ideal locations for the conferences in order to
minimize the footprint.  \ifopinions We draw in Section~\ref{sec:opinions}
concrete recommendation to the SIGPLAN conference organizers based on this
analyzes.  \fi Finally, we advertise for the use of the open source tool we
developed to conduct our analyzes in Section~\ref{sec:software}, hoping that
other communities might piggyback on our effort to conduct similar studies.

