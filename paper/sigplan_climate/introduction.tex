\section{Introduction}

Given the existential threat of global warming, it is incumbent on
individuals and organizations to evaluate the carbon emissions
associated with their activities and find ways to reduce them.  For many
academic researchers, these emissions will overwhelmingly come from air
travel, especially to international conferences.

This observation raises a number of questions about how to organize our
professional activities so as to maximize progress while minimizing
emissions.  Should SIGPLAN conference locations be chosen to minimize their
carbon impact? If
so, how? Should we move toward co-locating conferences? Or, on the contrary,
should some conferences be split into regional meetings or held
simultaneously at two sites on different continents?
Should we continue holding some conferences entirely virtually,
post Covid?
% \bcp{This is a
%   good question, of course, but I'm not sure it is addressed at all by the
%   data we have.}
% \yz{My rationale was that all these questions require hard numbers to ponder
%   correctly, which is the object of this paper. None of these questions should
%   really find an answer here, should it?}

To ground discussions about the decisions and compromises that the
scientific community may collectively wish to undertake, at least three
main sorts of data seem useful.
\begin{itemize}
\item The estimated emissions of past conferences.
\item The geographical distribution of participants to conferences.
\item The overlap in participation between various conferences.
\end{itemize}

We outline the results of a preliminary analysis of the past several
years of registration data for four of the main SIGPLAN conferences.  We
hope this effort can serve as a basis both for debates about concrete
measures and for larger and more comprehensive studies.

After briefly describing our dataset in Section~\ref{sec:dataset}, we
present estimates of the individual footprints of each conference in
Section~\ref{sec:footprint}.  In Section~\ref{sec:community}, the core of
our analysis, we derive several statistics about the geographical
distribution of participants and their habits of
cross-participation---across years and across conferences---arguing that
these data are correlated to the footprint.  We then present in
Section~\ref{sec:speculate} a speculative experiment aiming to estimate
``ideal locations'' for past conferences in order to minimize their
footprints.  \ifopinions In Section~\ref{sec:opinions} we draw some concrete
recommendations for future conference organizers based on these analyses.
\fi Finally, in Section~\ref{sec:software}, we outline the open-source tool
we developed to conduct our analyses, in hopes that other communities might
piggyback on our efforts to conduct their own similar studies.
