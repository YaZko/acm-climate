\documentclass{scrartcl}

\usepackage[english]{babel}
\usepackage[utf8x]{inputenc}

\usepackage{caption}
\usepackage{subcaption}

\usepackage{amsfonts} % mathbb
\usepackage{amsthm}   % environment proof
\usepackage{stmaryrd} % semantical brackets

\usepackage{siunitx} 

\usepackage[hidelinks]{hyperref}
\usepackage{xcolor}
\newif\ifcomments\commentstrue   %% include author discussion

\usepackage{glossaries}
\usepackage{csvsimple}

\ifcomments
\newcommand{\yz}[1]{\textcolor{blue}{{[YZ:~#1]}}}
\newcommand{\bcp}[1]{\textcolor{red}{{[BCP:~#1]}}}
\else
\newcommand{\yz}[1]{}
\newcommand{\bcp}[1]{}
\fi

\newcommand{\python}{\texttt{Python 3}}
\newcommand{\gaz}{\si{\kilogram\of{CO_2e}}}
\newcommand{\gazunit}{\si{\kilogram\of{CO_2e~ per~ passenger}}}
\newcommand{\gazunitbis}{\si{\tonne\of{CO_2e}}}

\newcommand{\event}{event} % What's the right term to designate a specific iteration of a conference?
\newcommand{\conf}{conference} 

\newtheorem{obs}{Observation}
\newtheorem{recommend}{Recommendation}

\title{Engaging with Climate Change: grounding in data possible changes for SIGPLAN}

\begin{document}

\maketitle

\bcp{Probably better not to use the same title as the earlier report.}

\section{Introduction}

In face of the global warming threat, we are responsible to evaluate the carbon
emissions entailed by our activity, and seek for ways to reduce this footprint.
It is well-known that the overwhelming factor of emission in research is air travel,
notably due to international conferences.

With this perspective in mind, we may be attracted to ponder several questions
about our activity.
\begin{itemize}
% \item Should we organize the offsetting of the carbon cost of conferences? If
%   so, what should be the amount of carbon we offset, at what price, and to whom
%   should the cost be attributed?
%   \bcp{IMHO, this is a topic for a separate paper.}
%   \yz{I had in mind that the numbers could legitimate the amount we ask for, but I see your point, we can certainly ditch this section. I commented it out for now}
\item Should SIGPLAN conference locations be chosen to minimize their
carbon impact?  
\item Should some conferences be co-located?
\item Should some conferences be split into two remote sites?
\item Should some conferences be held entirely virtually?  \bcp{This is a
  good question, of course, but I'm not sure it is addressed at all by the
  data we have.}
\end{itemize}

Such decisions have various impacts on our activity, and may have drawbacks in
terms of the social and professional benefits of conferences. 
In order to ground discussions about the compromises we wish to undertake, we consider
three main classes of data whose analysis may guide the decisions of the community.
\begin{itemize}
\item What has been the emissions of past conferences?
\item What is the geographical distribution of participants to conferences?
\item What is the overlapping in participation that exists between various conferences?
\end{itemize}

We propose in this paper one such analysis that we hope can be seen as a basis both for
debates about concrete measures, as well as the basis for more involved data analyses.

Section~\ref{sec:dataset} and \ref{sec:methodo} describe respectively the
analyzed data set and the methodology followed. The aggregated data
we derived is then exposed through Section~\ref{sec:data}: we point a few
observations of interests, but aim to remain neutral through this exposition.
Finally, we offer the conclusions we derive from the data and the measures we
correspondingly advice the ACM to take through Section~\ref{sec:opinions}.

\section{The data set}
\label{sec:dataset}

The currently available data spans over four conferences, all members of
SIGPLAN: POPL, PLDI, ICFP and SPLASH. Data ranges from 2009 to 2019, with a few
missing data points.

For each instance of each conference, the available information is the location
(city, state if relevant and country) the conference took place in, as well as
the list of all participants paired with their location of origin. The identity
of participants is hashed, leading to anonymization but retaining the
possibility to study cross participation.

The resulting dataset contains in total 4 different conferences, 33
instances of conferences for a total of 16374 participants.

\section{Evaluation of the carbon cost: methodology}
\label{sec:methodo}

All data are processed using \python.

Informal named locations manually provided by participant are mapped to their
ISO designation using the \texttt{pycountry} library.
Once this is done, these named locations are converted to GPS
locations using the \texttt{geopy} library, that provides a straightforward api
to this end.

An open source data-base of airports and their locations is then used to
manually find the closest one to each city of concern, among those classified as
\emph{medium} or \emph{large}.

Distances in kilometers between airports are then computed between GPS locations
once again using the \texttt{geopy} library. They use the geodesic distance
(shortest distance for an ellipsoidal model of the Earth) with a model providing
precisions that are several orders more precise than we need.

At this point, we therefore know for each participant to a conference the
distance separating the two airports between which he is most likely to have
commuted. It is relevant to keep in mind the following assumptions that have
been made to reach this stage:
\begin{itemize}
\item we assume that \emph{all} participant travelled by plane;
\item we assume that all flights are direct flights;
\item we assume that the geodesic distance is the one taken by planes.
\end{itemize}

Estimating the errors introduced by these hypotheses and refining them would be
a valuable work. However, for this first work that mainly aim at a relative
evaluation of different structural changes in our activities, we believe those
to be workable hypotheses.

Remains now to convert this distance to an emission in \gaz. We use to this end
a standard model introduced as part of the \texttt{DEFRA 16} report on
Greenhouse gas
\footnote{\url{https://www.gov.uk/government/publications/greenhouse-gas-reporting-conversion-factors-2016}}
\footnote{\url{https://co2calculator.acm.org/methodology.pdf}}
conducted by the British Government.

The model distinguishes three classes of flight, depending on their length:
short, medium or long hauls. Each category is associated with a linear
coefficient relating directly the distance of travel to the amount of \gaz
emitted. We make the assumptions, anecdotally observed, that researchers all
fly in economy class.

A second linear coefficient, identical for all flights, is added to
account for radiative forcing\bcp{I think we (I) have not been using this
  term quite correctly: we should really be saying ``radiative forcing
  index''---i.e., here, the coefficient is accounting for the {\em
    difference} in radiative forcing between the same emissions at ground
  level vs. high in the atmosphere}, i.e. an estimation of the impact that the
specific gases emitted in the specific context of aviation has on global
warming.

We therefore obtain the following piece-wise linear model of emissions for a flight covering $d$ kms:
\begin{itemize}
\item $1.891 * 0.14735 * d$ \gazunit if $d < 785$
\item $1.891 * 0.08728 * d$ \gazunit if $785 \leq d < 3700$
\item $1.891 * 0.077610 * d$ \gazunit if $3700\leq d  $
\end{itemize}

It should be noted that experiments with other models show significant variance
in absolute value, but resilience in relative values.\bcp{Maybe worth
  showing some numbers justifying these statements?}\yz{I agree, will
  do}\bcp{Assuming that we can get our numbers to agree with CoolEffect's,
  we could also mention this!} Once again, refining the
model would hence be a valuable work, but using this simple standard and
well-established one appears appropriate to draw conclusion in terms of
\emph{relative} impact of different measures.

\section{Data analysis}
\label{sec:data}

Through this section, we present the results of our data analysis in a as
neutral way as possible. We shall point out phenomena that came out as a
surprise to us, but defer opinionated observations and practical conclusions to
Section~\ref{sec:opinions}.

\subsection{Raw data}

\begin{figure}
\begin{tabular}{|l|l|c|c|c|}
  \hline%
  \bfseries event & \bfseries location & \bfseries nb. participants & \bfseries total cost & \bfseries average cost 
\csvreader[head to column names]{../output/acm/footprint_confs.csv}{}%
{\\\conf\ \year & \location & \csvcoliv & \csvcolv & \csvcolvi}%
\\\hline
\end{tabular}
\caption{For each \event: location, number of participants and carbon cost, total and average per participant, in \gazunitbis,}
\label{fig:RawData}
\end{figure}

Figure~\ref{fig:RawData} depicts the total and average carbon cost per participant of
all conferences analyzed. This cost is estimated in terms of \gazunitbis of emissions.
The main data of interest is naturally the last column depicting the average cost per participant.

This raw data already presents a very obvious and interesting phenomenon: a
factor two separates the lowest average carbon cost (PLDI'18 at 0.9\gazunitbis)
from the highest one (ICFP'16 at 1.93\gazunitbis).
Understanding the underlying rationals for such variations may be a promising
angle to reduce our emissions without restructuring fundamentally our activity.

Since emissions are strongly correlated to the distance traveled, deriving
demographic data may shine some light.

\subsection{Demographics}
\label{subsec:demo}

\begin{figure}
  \csvautotabular{../output/acm/demographic.csv}
  \caption{For each \event, continent in which it took place and distribution of
    each continent by origin of participants. The final column indicates the
    portion of participants that traveled from the same continent the
    conference took place in.}
  \label{fig:demo-raw}
\end{figure}

Demographic data are a key resource to understand the carbon cost of attendance.
Figure~\ref{fig:demo-raw} shows, for each \event, the
continent it took place in, the distribution of attendance per continent, and
the percentage of attendants originating from the same continent as the one it
took place in. To a first approximation, maximizing this last parameter is a fair
heuristic to minimize the carbon cost without any impact on the organization of
the conference itself. One may already notice that in the current conjecture, in
sheer question of volume, attendance is essentially divided between Europe and
North America primarily, Asia secondarily.

\begin{figure}
  \csvautotabular{../output/acm/demographic_per_conf.csv}
\caption{For each kind of conference, distribution of participants per continent of origin}
\label{fig:demo-conf}
\end{figure}

A natural idea suggested by the desire to increase the percentage of attendance
taking place locally at the scale of continents is to look for a core group of
recurrent attendees that would constitutes a model of the distribution of the
attendees of a given conference. To investigate this idea, Figure~\ref{fig:demo-conf}
depicts an aggregated information of the previous data: for each 
\conf, the average distribution of attendance per continent. From this
table, one can observe the relative cultural anchor of each conference into a
specific continent. Most notably, PLDI and SPLASH appear to be very North
American-centric, while ICFP's core community seems to have a quite strong
anchor in Europe as well.

This aggregation of data however intrinsically rests upon the assumption of a uniform
community attending each conference every year. A closer inspection of
Figure~\ref{fig:demo-raw} easily convinces us of the erroneous dimension of this
hypothesis. Most strikingly, Asian participation during POPL '15, ICFP '11 and
ICFP '16, events that took place on the Asian continent, is significantly higher
than usual: there appears to be a strong locality phenomenon. Crossing the data with
Figure~\ref{fig:demo-conf} one can also notice that the only time SPLASH took
place in Europe turned out to be the least expensive edition, challenging our previous
observation that the conference appears to be mostly north American-centric.

\begin{figure}
  \csvautotabular{../output/acm/demographic_delta.csv}
\caption{Geographical distribution of participation conditioned by the location of the \event}
\label{fig:local_effect}
\end{figure}

Figure~\ref{fig:local_effect} attempts to measure this locality effect. The
table depicts, all conferences being considered at once, the geographical
distribution of attendance conditioned by the geographical location of the
\event. The Asian phenomenon previously hinted at is here extremely
apparent: while overall on average, 10.9\% of the participants come from Asia,
this number is roughly multiplied by a factor 4 when the \event takes place in Asia --
without any significant drop in total volume of attendance that could indirectly bump
the percentage.

But interestingly, this phenomenon also exists in the case of Europe (+22.29\%
deviation to the average) and North America (+12.15\% deviation to the average).
Despite their name, international conferences appear to exhibit a fairly strong
local component.

\subsection{Attendance overlaps}
\label{subsec:overlap}

Section~\ref{subsec:demo}, through the study of the demographic distribution of
attendance, has suggested the existence of local communities that only
partake in conferences when they take place close to their place of residency.
One can conversely look for groups of regular attendees, that participate to a given \conf
regardless of the location it is held in.

\begin{figure}
\centering
     \begin{subfigure}[b]{0.4\textwidth}
       \centering
       \csvautotabular{../output/acm/overlap_intra_conf_POPL.csv}
       \caption{Case of POPL}
     \end{subfigure}
     \hfill
     \begin{subfigure}[b]{0.4\textwidth}
       \centering
       \csvautotabular{../output/acm/overlap_intra_conf_ICFP.csv}
       \caption{Case of ICFP}
    \end{subfigure}

     \caption{For any two years, percentage of overlap in attendance at the corresponding editions of a conference (part 1)}
     \label{fig:overlap-conf-alpha}
\end{figure}

\begin{figure}
\centering
     \begin{subfigure}[b]{0.4\textwidth}
       \centering
       \csvautotabular{../output/acm/overlap_intra_conf_PLDI.csv}
       \caption{Case of PLDI}
     \end{subfigure}
     \begin{subfigure}[b]{0.4\textwidth}
       \centering
       \csvautotabular{../output/acm/overlap_intra_conf_SPLASH.csv}
       \caption{Case of SPLASH}
     \end{subfigure}

     \caption{For any two years, percentage of overlap in attendance at the corresponding editions of a conference (part 2)}
     \label{fig:overlap-conf-beta}
\end{figure}

Figures~\ref{fig:overlap-conf-alpha}~and~\ref{fig:overlap-conf-beta} present one
way to gather intuition about the reality and size of this phenomenon. For a
given conference at a time, we depicts, for any pair of years, the percentage of
attendees that participated in both events. In particular for two consecutive
years, this number oscillate between 15\% and 30\%. The vast majority of
participants of a conference were therefore not part of the previous edition.

\yz{TODO: This data is not striking enough and takes a lot of space. To aggregate
  more, or ditch in favor of the following tables only}\bcp{IMO it is worth
  keeping.  But maybe it could be moved later?}

\begin{figure}
  \centering
  \begin{subfigure}[b]{0.3\textwidth}
    \centering
    \csvautotabular{../output/acm/overlap_cross_conf_ICFP_POPL.csv}
    \caption{POPL and ICFP}
  \end{subfigure}
  \begin{subfigure}[b]{0.3\textwidth}
    \centering
    \csvautotabular{../output/acm/overlap_cross_conf_POPL_PLDI.csv}
    \caption{POPL and PLDI}
  \end{subfigure}
  \begin{subfigure}[b]{0.3\textwidth}
    \centering
    \csvautotabular{../output/acm/overlap_cross_conf_POPL_SPLASH.csv}
    \caption{POPL and SPLASH}
  \end{subfigure}
  \begin{subfigure}[b]{0.3\textwidth}
    \centering
    \csvautotabular{../output/acm/overlap_cross_conf_ICFP_PLDI.csv}
    \caption{ICFP and PLDI}
  \end{subfigure}
  \begin{subfigure}[b]{0.3\textwidth}
    \centering
    \csvautotabular{../output/acm/overlap_cross_conf_ICFP_SPLASH.csv}
    \caption{ICFP and SPLASH}
  \end{subfigure}
  \begin{subfigure}[b]{0.3\textwidth}
    \centering
    \csvautotabular{../output/acm/overlap_cross_conf_PLDI_SPLASH.csv}
    \caption{PLDI and SPLASH}
  \end{subfigure}
   \caption{For every year, overlap in attendance between the events of two
     different conferences\bcp{Maybe we could also have an aggregate count
       for each pair, showing how many people {\em ever} went to both (even
       in different years).}}
  \label{fig:overlap-cross}
\end{figure}

Arguably more useful for practical purposes is to evaluate the overlap 
between the events for the same year of two different conferences.
Figure~\ref{fig:overlap-cross} depicts this data for any pairing of the four
conferences considered. The overlap is strikingly low for most conferences.

\begin{figure}
  \csvautotabular{../output/acm/number_of_participations.csv}
\caption{Overall and for each conference, the average number of instances a
  participant has taken part of, and the percentage of them that has
  attended at least $k$ instances, for $k\in\llbracket 2 \dots 5
  \rrbracket$. Remark: the means and percentages are here computed with
  respect to \emph{unique} participants.  \bcp{I really like this table.}}
\label{fig:reccurent}
\end{figure}

\begin{figure}
  \centering
  \begin{subfigure}[b]{0.4\textwidth}
    \centering
    \csvautotabular{../output/acm/old_timer_POPL.csv}
    \caption{Case of POPL}
  \end{subfigure}
  \begin{subfigure}[b]{0.4\textwidth}
    \centering
    \csvautotabular{../output/acm/old_timer_ICFP.csv}
    \caption{Case of ICFP}
  \end{subfigure}
  \\
  \begin{subfigure}[b]{0.4\textwidth}
    \centering
    \csvautotabular{../output/acm/old_timer_PLDI.csv}
    \caption{Case of PLDI}
  \end{subfigure}
  \begin{subfigure}[b]{0.4\textwidth}
    \centering
    \csvautotabular{../output/acm/old_timer_SPLASH.csv}
    \caption{Case of SPLASH}
  \end{subfigure}
  \caption{For each conference, percentage of participants that have been
    part of a previous edition of the same conference \bcp{Is this sort of
      the same information as Figure 8?}}
  \label{fig:old-timers}
\end{figure}

Finally, Figure~\ref{fig:reccurent} and \ref{fig:old-timers} offer two different views on reccurent participations. Figure~\ref{fig:reccurent} represents respectively for the whole dataset and for each conference individually the average number of editions an average participant has been part of, as well as the percentage of participants that has been part of at least some amount of times. We immediately remark that no less than 75\% of unique participants has been part of a single edition.
Figure~\ref{fig:old-timers} represents for each instance of each conference the percentage of participants that has participated in a previous instance of the conference, among our dataset.

\subsection{A naive, retrospective, optimal choice}

We have observed that the location an \event takes place in significantly
impacts the distribution of origin of its participants. However, dismissing
temporarily this factor to consider what could have been the cheapest location
for past conferences can be an illuminating exercise.

To this end, we chose a fixed number of locations that we believe to be
representative and spread across the relevant parts of the globe for our
concern: Paris, Edinburgh, Boston, Los Angeles, Vancouver, Tokyo, Beijing and
Mumbai. We then reprocessed the dataset to look for the location that would have
led to the lowest carbon footprint, once again assuming that it would not have
changed the set of participants.

\begin{figure}
  \begin{tabular}{|l|l|c|c|c|c|}
    \hline%
    \bfseries event & \bfseries orig. loc. & \bfseries orig. cost & \bfseries best loc. & \bfseries best cost & \bfseries saved
    \csvreader[head to column names]{../output/acm/optimals.csv}{}%
              {\\\conf\ \year & \csvcoliii & \csvcoliv & \csvcolv & \csvcolvi & \csvcolvii}%
              \\\hline
  \end{tabular}
  \caption{For each \event, depicts the location among Paris, Edinburgh,
    Boston, Los Angeles, Vancouver, Tokyo, Beijing and Mumbai that would
    have led to the lowest carbon footprint. We indicate 'SAME' when the
    actually location is the best, and show in the final column the amount
    of \gazunitbis that it would have saved. \bcp{Might be more informative
      to replace SAME with the name of the city, in parens or something to
      indicate the sameness.}}
  \label{fig:optimal}
\end{figure}


Figure~\ref{fig:optimal} depicts the resulting data: for each \event, the best location
and the average \gazunitbis it would have saved. We observe that in the majority
of the events, the locality effect is strong enough that the optimal takes place on
the original continent. However it is striking to see how often the East Coast turns
out to be the cheapest destination. In particular it appears to be preferable to
the West Coast in most cases, even despite the underlying locality effect that
is ignored here.

\section{Opinionated interpretation}
\label{sec:opinions}

The data analysis conducted through Section~\ref{sec:data} has put in evidence
several phenomena that we review here.

\bcp{These observations do not seem all that opinionated.  Maybe move them
  earlier, with the figures they refer to?}

The first one is a straight read of the carbon footprint of conferences. 
\begin{obs}
If as is well known the carbon footprint of
conferences due to air travel is significant, the specific number
varies from an \event to another by up to a factor of 2.
\label{obs:footprint}
\end{obs}

The more specific analysis of demographic data, in Section~\ref{subsec:demo},
gives a sense of the rough origin of participants:

\begin{obs}
  The vast majority of participants are split between North America and Europe,
  Asia to a much lesser degree. SPLASH and PLDI are strongly anchored in North
  America, ICFP and POPL fairly equally split between North America and Europe.
  \label{obs:dist-naive}
\end{obs}

This distribution however turns out to be \emph{strongly} dependent on the
location of the \event.

\begin{obs}
  There is a major ``locality" effect: it is both true that locality attract
  new participants, and distance repels some participants.
  \label{obs:locality}
\end{obs}

The extent of this phenomenon should not be underestimated. The example of the
few conferences having taken place in Asia underlines the size of the community
that we fail to encounter otherwise. Also interesting, the surprisingly cheap
footprint of the only time SPLASH took place in Europe may suggest that the
North American anchor of the conference is not intrinsic, but rather a
self-fulfilling prophecy.

Section~\ref{subsec:overlap} provides insights into the expected ``core" community
that would attend most editions of a given conference by analyzing the overlap
in participations that actually took place.

\begin{obs}
  Temporal overlap is moderate: roughly a quarter of attendants were present
  the year before at the same conference.
  \label{obs:overlap-temp}
\end{obs}

\begin{obs}
  Cross-conference overlap is low: the tightest pairing sees slightly over 10\%
  of common attendance for a given year.  \bcp{More aggregated numbers would
  also be interesting, since I suspect that there are quite a few people
  that only go to one big conference a year, but may alternate between
  (e.g.) POPL and PLDI.}
  \label{obs:overlap-cross}
\end{obs}

Additionally, a significant portion of attendees are ``one timer", or close.

\begin{obs}
  The average amount of conferences a participant has been part of is extremely low: 1.52.
  Less than 4\% of unique participants have been to more than five events
  among our dataset\bcp{Is that an accurate way of saying it?  I thought it
    was ``have been to five instances of the same conference.''  (The other
    number would be interesting too---how many people went to 2,3,4,5
    instances of {\em any} of the conferences?}.
  Similarly, at any \event, more than half of the participants are experiencing this specific
  conference for the first time.
  \label{obs:old-timers}
\end{obs}

Lastly, despite the aforementioned locality effect that tends to narrow the
distribution of attendance around the locality the conference actually takes
place in, a crude search for lowest carbon footprint is highly instructive.

\begin{obs}
  Due to the locality effect, past data can act as a heuristic for a worst case
  distribution of attendance with respect to the objective function of
  minimizing the carbon footprint. Doing so most notably suggests that the East
  Coast should often be preferred to the West Coast.
  \label{obs:optim}
\end{obs}

\subsection{A mandatory estimate of the carbon footprint by the conference organizers}

Despite a modest amount of data at our disposition and the use of a rudimentary suit of
analyses, there is no ambiguity about the relative environmental impact the choice of
location to hold a conference in has. In particular, observation~\ref{obs:dist-naive}
suggests that even setting aside any restructuration of our activities, we can hope for
saving a factor 2 by being more acute when choosing destinations. Furthermore,
observation~\ref{obs:optim} emphasizes that even a naive distribution model already gives
us material to do better, while the more ambitious perspective to model the locality effect
that we discussed through this paper would allow us for even more efficient choices.

In this light, we consider it unacceptable to continue choosing locations of conferences
either blindly, or for its scenic value. We should ponder professional relevancy with
ecological imperative. We hence formulate the following simple recommendation, that shall
have no impact on our professional activity, save for the reduction of some leisuring side
product.

\begin{recommend}
It should be made part of the mandatory process of organization of SIGPLAN
conference to estimate the carbon footprint of the options considered, and
to take the results of this analysis into account to finalize the decision.
\end{recommend}

It shall be emphasized that we do not suggest by this to consider the destination
minimizing the carbon footprint as the systematic right choice. Concerns such as
rotating over different parts of the globe or naturally accounting for availability
of qualified universities to organize should remain of major concern. We merely
assess by this recommendation the need to bring carbon footprint into the constraint
system we seek to optimize.

\subsection{A short term experiment: bi-localized or tri-localized conferences}

A considerate choice of destination to organize conferences can lead to a
non-neglectible reduction of their carbon footprint. However, a reduction of
the scale required to match by 2050 the recommendation from the Accords de Paris
will require more drastic measures. More specifically, we need to reduce the cheer
number of flights our activity induces.

There is no denying that it will have an impact on our activity, some of which will
be negative. It is hence more than ever of importance to take a reasoned approach
allowing us to balance optimization of quantitative measures, such as reducing the
carbon footprint, with qualitative imperative, such as maintaining the ideal of
an international, borderless, scientific research.

Interestingly, this novel requirement leads us to pay attention to data that may also be
relevant to our activity beyond the question of carbon footprint. These should be
taken into consideration as well while seeking a lasting restructuration of our
activity. In particular, it is implicit to assume that conferences have to be
geographically international to gather communities of researchers from all over the
world. However, observation~\ref{obs:locality} challenges strongly this intuition:
despite any claim a conference may have, the very fact that it is hold each year in
a single place on Earth rules out a vast amount of international researchers. It is most
striking with respect to programming language communities from Asia, but seems to be
true for Europeans desiring to partake in SPLASH as well for instance.

This statement is also backed up by evidence against its complement: the idea of a core
group of researchers making the essential of all editions of their favorite conference,
while not completely incorrect, is vastly overestimated. Observations~\ref{obs:old-timers}
most notably makes it very clear.

This analysis leads us to push toward strong considerations for experimenting a more
ambitious way to save carbon: giving up on the uniqueness of location of conferences and
experimenting with bi-localized or tri-localized conferences.

\begin{recommend}
  Some conferences should experiment a bi-localized format. Typically, POPL could for
  instance be held simultaneously held in Boston and Paris. A day would span over 12 hours
  instead of the usual 8. The four hours intersecting would be held simultaneously on
  both sites via visioconference. The eight other hours would be retransmitted live and
  have simple support for questions as is already put in practice.\\
  If the initial experiments go well, we recommend a progressive shift toward this
  format becoming the norm, and consideration for a third site.
\end{recommend}

We argue naturally that this change would be a truly ambitious measure to reduce
significantly the carbon footprint of conferences. But furthermore, we believe that
it would also enhance the international dimension of the conference: following
the locality effect, this would most certainly lead to an increase in participation.

A natural opposition would be to state that it is unreasonably to ask for researchers to
follow twelve hours a day of conferences, and that they would therefore miss part of the
talks. We do not deny this fact, but points out that it is already largely the case,
most conferences having two, if not three tracks in parallel.

Yet, it should not be brushed aside that this would remove some precious
physical interactions between researches from different continents. We
nonetheless argue that making these interactions the systematic default at conferences
is an historical incident. Such a restructuration of our activity would probably
be accompanied by an increase in visit to other laboratories. But that would shift
these interactions from the current situation that put hundreds of researchers in the
same building so that extremely small groups get to meet, to a more sensible ``meeting
by need" organization.

\subsection{A long term need: entirely virtualized conferences}

Bi-localized conferences strike a compromise. On the long run, research, as all
activities, shall however ambition to be entirely carbon-free. This ambitious
goal has already been embraced by some conferences\footnote{https://conference.opensimulator.org/2018/}
and seminars\footnote{https://sites.google.com/site/plustcs/}.

While the currently existing cases are either fairly experimental, or of much more
modest size than a conference such as the ones organized by SIGPLAN, they report
encouraging results. As such, we encourage experiments aiming to develop further
these techniques, and bring the cultural change they entail incrementally among
our community.

\begin{recommend}
  We recommend to conduct experiments toward the development of fully virtual
  conferences as a mean to reach a fully sustainable activity by the horizon 2050
  at the latest.
\end{recommend}

%% \subsection{Offsetting}

%% Striving for carbon neutrality, the spontaneous reaction is to reduce emissions.
%% Carbon offsetting works at the other end of the scale: by increasing the rate of
%% absorption of carbon from the atmosphere, through so-called carbon wells, we can
%% get away with a higher rate of emissions.

%% The best known carbon well is naturally trees themselves: a caricatural
%% description of carbon offsetting is hence to plant a sufficient amount of trees
%% to compensate for our emissions. In practice, there are numerous vastly
%% different carbon wells, whose efficiency and cost differs vastly, or even often
%% are extremely hard to evaluate.

%% \yz{Refer to the sigplan paper on the topic}

%% The danger of carbon offsets is to mistakenly interpret them as a magical
%% bullet: carbon wells have margin to grow, but in no way can get anyway near able
%% to support our current rate of emission. They are however an available resource
%% that should be taken advantage of.

%% They also offer the advantage to interface easily with the market: they are an
%% unregulated mechanism allowing to put a price on carbon emissions, and hence if
%% combined with legislative enforcement of carbon neutrality, could let the market
%% sort the problem of allocating available carbon emissions.

%% They are at the moment not nearly nature enough: in particular, the current
%% price they set on carbon vastly understimate the estimated social cost of the
%% same emission. Additionally, companies have at the moment mainly altruistic or
%% PR insentives to use them, but very few coercive ones.

%% However, they have the significant advantage to be immediately trivially
%% implementable as a temporary solution while moving toward a restructuration of
%% the activity to reduce the emissions. For this reason, we recommend their
%% integration in conferences.

%% \yz{Talk about already existing first experiments}

%% Doing so requires to fix the amount of emissions we wish to offset, as well as
%% either a price per ton or leaving to the attendant to offset the emissions
%% themselves.

%% The data analyzed in this study offers us a way to choose a realistic answer
%% to the first question. It however raises the following choices:

%% \begin{enumerate}
%% \item Should we average over all of SIGPLAN? Previous editions of the same conference? Build a model of attendance and estimate for the incoming year?
%% \item Should we distribute this cost homogeneously, or should we bill each participant based on the distance travelled to encourage participation in conferences closed to them?
%% \item Should this tax be mandatory? Optional checked in by default? Optional check-out by default? Mandatory, but equip the conference with a fund available on demands to cover it? 
%% \end{enumerate}


\end{document}
